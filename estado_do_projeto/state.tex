Para identificar o estado atual do projecto, realizou-se este documento no qual são listadas as funcionalidades/requisitos a implementar, e se indica o grau de completude de cada uma.

Na tabela seguinte são apresentadas as funcionalidades ou requisitos funcionais planeados na fase inicial deste projecto, e é indicado, para cada uma deles, se foram implementados com sucesso.

\begin{center}
    \RowStretch{1.3}
\begin{longtable}{@{}lp{0.3\textwidth}cp{0.3\textwidth}@{}}
    \toprule \#  & Funcionalidade    & Progresso    & Comentários \\ \midrule
    \endfirsthead
    \toprule \#  & Funcionalidade    & Progresso    & Comentários \\ \midrule
    \endhead
    \bottomrule
    \caption{Caption}\label{tab:func}\\%
    \endfoot
    \bottomrule
    \caption[]{(continuação)}\\%
    \endlastfoot
    1   & Criar conta de utilizador.  & Concluído  &  \\
    2   & Editar conta de utilizador.  & Concluído  &  \\
    3   & Associar conta de serviço de gestão de dinheiro.  & Concluído  &  \\
    4   & Consultar histórico de gastos.  & 100  &  \\
    5   & Filtrar histórico de gastos por data.  & 100  &  \\
    6   & Filtrar histórico de gastos por localização.  & 0  & As funcionalidades relacionadas com geolocalização foram consideradas menos importantes e não foram implementadas por falta de tempo. \\
    7   & Consultar estado de conta.  & 100  &  \\
    8   & Definir limites de utilização do dinheiro em ordem ao tempo.  & 100  &  \\
    9   & Definir limites de utilização do dinheiro em ordem à localização.  & 0  & As funcionalidades relacionadas com geolocalização foram consideradas menos importantes e não foram implementadas por falta de tempo. \\
    10  & Definir regras de notificação simples.  & 100  &  \\
    11  & Definir regras de notificação complexas.  & 0  & Numa fase inicial, apenas foram implementadas notificações para situações comuns, mas no futuro irá ser implementado um motor de regras de notificação mais flexível. \\
    12  & Consultar notificações.  & 100  &  \\
    13  & Efetuar pedidos de acesso a quantias de dinheiro.  & 100  &  \\
    14  & Receber e responder a pedidos de acesso a quantias de dinheiro.  & 100  &  \\
    15  & Autorizar espontaneamente o acesso a quantias de dinheiro extraordinárias.  & 100  &  \\
    16  & Efetuar pagamentos a comerciantes.  & 0  &  \\
    17  & Transferir dinheiro para outros utilizadores.  & 0  &  \\
    18  & Definir metas de poupança.  & 100  & 
\end{longtable}
\end{center}

O que podemos retirar deste documento é que a grande maioria das funcionalidades foram implementadas com sucesso.
As funcionalidades que não foram completamente implementadas, ou foram identificadas como menos importantes, como as funcionalidades relativas à geolocalização, ou são consideravelmente complexas e a sua implementação foi adiada para uma fase posterior à do âmbito da unidade curricular em que este projecto se insere, para lhe dedicar o tempo necessário que a sua correcta implementação requer.
Com isto pode-se concluir que a solução final está practicamente funcional, e portanto este projecto pode ser considerado bem-sucedido.
