Projetos de dimensões consideráveis, tais como os projetos do âmbito desta unidade curricular, são complexos, já que envolvem muitos colaboradores e muitas fases de desenvolvimento. É portanto inevitável ocorrerem problemas e contratempos com o decorrer do projeto.
Para garantir portanto o sucesso de um projeto, é necessário fazer um planeamento rigoroso e detalhado, descrevendo as várias fases de desenvolvimento, a calendarização das tarefas e das entregas, e a implementação ou utilização de métodos de organização, para que os colaboradores do projeto conseguirem trabalhar todos eficientemente e em harmonia. É também necessário fazer uma análise de potenciais riscos e possíveis soluções.
Este documento tem o propósito de expor os métodos e os resultados do planeamento nos aspetos acima descritos.

\subsection*{Metodologia de Desenvolvimento}
Para a gestão deste projeto, usamos uma metodologia de desenvolvimento ágil, baseada em Scrum. Esta metodologia baseia-se no desenvolvimento parcelado em intervalos de tempo curtos, em que certas funcionalidades ou porções do projeto devem ficar completamente implementadas, evitando assim executar sequencialmente as várias fases associadas a um projeto, tal como a análise de requisitos, o desenvolvimento em si, etc, passando estas fases a serem executadas ciclicamente nesses intervalos de tempo curtos. Esta metodologia também facilita a interação entre os elementos do grupo, promovendo a capacidade de organização individual e como um todo.
O facto de esta metodologia requerir a execução contínua de cada um dos seus processos, nomeadamente a análise de requisitos, implica uma melhor capacidade de adaptação a situações que requeiram alterações, tais como uma mudança de requisitos por parte dos Stakeholders ou impossibilidades técnicas ou outros riscos inesperados.

\subsection*{Controlo de Tarefas}
Para a gestão e o controlo de tarefas, foi usada a plataforma Redmine, por sugestão da equipa docente. Nesta foram definidas tarefas de grande dimensão, para as quais os vários sub-grupos contribuíram. Para um controlo mais detalhado das várias tarefas, usamos a plataforma Trello, uma vez que esta é mais fácil de usar em conjunto com a metodologia Scrum, uma vez que fornece mais funcionalidades que esta metodologia requer, que são menos exploradas pela plataforma Redmine.
Para definir as maiores tarefas ou subdivisões de trabalho, usamos um diagrama Work Breakdown Structure (WBS), a partir do qual foi possível definir as várias etapas do projeto, definir equipas e alocá-las às várias tarefas e começar a fazer uso das plataformas referidas acima.

\addimg{img/diagram.png}

Com as tarefas definidas, passamos para a sua calendarização.
As tabelas seguintes apresentam as várias tarefas, as respetivas datas de conclusão e os momentos de entrega de \emph{deliverables}.

Na primeira fase, os \emph{deliverables} consistiam na Visão de Produto e no Documento de Requisitos. Para este propósito, após as tarefas necessárias para a organização da equipa e planeamento, investimos o nosso tempo a desenvolver estes documentos.

\begin{table}[H]\centering
    \RowStretch{1.2}
    \begin{tabular}{@{}p{0.4\textwidth}l@{}}\toprule
    Tarefa & Data de Conclusão \\
    \midrule
    Estabelecimento de ferramentas a utilizar para organização e comunicação.& 8 de Outubro de 2014 \\
    Definição de uma proposta inicial do produto. & 15 de Outubro de 2014 \\
    Apresentação inicial. & 15 de Outubro de 2014 \\
    Definição de equipas de trabalho. & 22 de Outubro de 2014 \\
    Definição e divisão de tarefas para a primeira fase. & 22 de Outubro de 2014 \\
    Definição da proposta do produto. & 9 de Novembro de 2014 \\
    Análise da concorrência. & 9 de Novembro de 2014 \\
    Entrevistar e partilhar questionários com potenciais clientes. & 9 de Novembro de 2014 \\
    Elaboração dos documentos para a entrega. & 14 de Novembro de 2014 \\
    Primeira entrega. & 14 de Novembro de 2014 \\
    \bottomrule
    \end{tabular}
    \caption{Support of paid tools for coding standards.}
    \label{tab:tool_support}
\end{table}

A entrega seguinte é a entrega final do projeto.
Esta etapa começa-se por uma fase inicial de modelação do sistema. Depois seria dedicado algum tempo de investigação e treino sobre as tecnologias e ferramentas a utilizar para o desenvolvimento, seguido do desenvolvimento propriamente dito. Como o desenvolvimento seria realizado por duas equipas separadas que depois conciliariam as duas partes do sistema (frontend e backend), após o desenvolvimento independente destas duas partes, unifica-se o sistema, e procede-se aos testes do mesmo, e em paralelo atualiza-se a documentação, para ter tudo preparado para a entrega final.

\begin{table}[H]\centering
    \RowStretch{1.2}
    \begin{tabular}{@{}p{0.4\textwidth}l@{}}\toprule
    Tarefa & Data de Conclusão \\
    \midrule
    Modelação do sistema. & 26 de Novembro de 2014 \\
    Investigação e treino em tecnologias. & 10 de Novembro de 2014 \\
    Desenvolvimento do front-end. & 14 de Janeiro de 2015 \\
    Desenvolvimento do back-end. & 14 de Janeiro de 2015 \\
    Unificação do sistema. & 21 de Janeiro de 2015 \\
    Elaboração de testes. & 25 de Janeiro de 2015 \\
    Elaboração dos documentos para a entrega final. & 1 de Fevereiro de 2015 \\
    Entrega final. & 4 de Fevereiro de 2015 \\
    Apresentação final. & 13 de Fevereiro de 2015 \\
    \bottomrule
    \end{tabular}
    \caption{Support of paid tools for coding standards.}
    \label{tab:tool_support}
\end{table}

\subsection*{Gestão de Recursos Humanos}
Para promover o bom funcionamento do grupo e portanto o sucesso do projeto, foram usados vários processos para gerir os elementos do grupo.
Atempadamente, foram identificadas as competências técnicas, entre outras, de cada elemento do grupo, para uma melhor distribuição das tarefas; o grupo sub-dividiu-se em várias equipas, para abordarem tarefas diferentes, e cada um destas equipas tinha um elemento elegido como líder, com a responsabilidade de gerir os elementos dentro da equipa, monitorizando-os para garantir o seu desempenho.

As posições estáticas ao longo do projeto foram as seguintes:
\begin{itemize}
    \item Líder do projeto: Fernando Nogueira
    \item Sub-líder do projeto: Benjamim Sonntag
    \item Restantes elementos:
        \begin{itemize}
            \item André Santos
            \item Bruno Silva
            \item Marco Pereira
            \item Miguel Carvalho
            \item Nelson Moutinho
            \item Rui Vilas Boas
            \item Sérgio Dias
        \end{itemize}
\end{itemize}

Durante a primeira fase formaram-se duas equipas:
\begin{itemize}
    \item Equipa 1:
        \begin{itemize}
        \item Líder:  Benjamim Sonntag
        \item Restantes elementos:
            \begin{itemize}
                \item André Santos
                \item Fernando Nogueira
                \item Nelson Moutinho
                \item Rui Vilas Boas
            \end{itemize}
        \end{itemize}
    \item Equipa 2:
        \begin{itemize}
            \item Líder: Sérgio Dias
            \item Restantes elementos: 
                \begin{itemize}
                    \item Bruno Silva
                    \item Marco Pereira
                    \item Miguel Carvalho
                \end{itemize}
        \end{itemize}
\end{itemize}

Durante a segunda fase formaram-se duas equipas:
\begin{itemize}
    \item Equipa do Backend:
        \begin{itemize}
            \item Líder: Benjamim Sonntag
            \item Restantes elementos:
                \begin{itemize}
                    \item Bruno Silva
                    \item Miguel Carvalho
                    \item Rui Vilas Boas
                \end{itemize}
        \end{itemize}
    \item Equipa do Frontend:
        \begin{itemize}
            \item Líder: Fernando Nogueira
            \item Restantes elementos:
                \begin{itemize}
                    \item André Santos
                    \item Marco Pereira
                    \item Nelson Moutinho
                    \item Sérgio Dias
                \end{itemize}
        \end{itemize}
\end{itemize}

\subsection*{Gestão de Comunicação}
Para garantir a comunicação entre todos os indivíduos envolvidos no projeto, foram tomadas várias medidas:
\begin{itemize}
    \item Reuniões semanais presenciais entre os elementos do grupo
    \item Reuniões semanais presenciais com os docentes
    \item Reuniões ocasionais não-presenciais entre os elementos do grupo
    \item Dialogo constante não-presencial entre os elementos do grupo
    \item Para a comunicação não presencial, foram utilizadas várias ferramentas online:
    \item Slack: permite a discussão entre elementos do grupo, dividido por salas conforme equipa a que cada elemento está alocado, ou relativo a certos assuntos
    \item Redmine: para a definição e controlo de tarefas
    \item Trello: para a definição e controlo de tarefas
    \item GitHub: para a gestão do código do programa, e feedback de erros encontrados
\end{itemize}

\subsection*{Gestão de Qualidade}
Para garantir a qualidade do projeto desenvolvido foram tomadas várias medidas:
\begin{itemize}
    \item Planeamento prévio
    \item Modelação
    \item Validações continuas
    \item Testes da solução implementada
\end{itemize}
